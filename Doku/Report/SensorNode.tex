%about the sensor node...

The sensor node provides the current state of the flower environment to the controller.
It is a low power device which guarantees long battery duration.

The sensors used are:
\begin{itemize}

\item Temperature sensor
It shows the temperature...

\item The second sensor is a moisture sensor. It consists of two parts, a resistance probe (YL-69) and a control board (YL-38). The probe will be put into the measured soil, while the board should stay dry. The Arduino nano has a 10 bit ADC, so the result read from the ADC is an integer between 0 and 1023.\\


The test showed that in a dry state the sensor shows 1023. Connecting the electrodes with a finger leads to a value around 1000.\\
Also in dry soil the readout is 1023, but as we pour water into the pot, it continuously decreases down to a few hundred. (I did not want to drown the flower, but I assume it goes to 0 if we put the probe into water...)\\

We might have to calibrate the moisture levels for different flower or soil types, but in general the sensor seems to do the job very well and also is usable with a XBee board because of its analog output. The only downside is that it's written in forums that after some weeks this type of sensor will be corroded, so we cleaned it as well as possible after the test. For developing and testing it is fine, in a real environment one might want to use another sensor like a capacitive one or with other electrode materials. Also turning off the sensor when not using it is a good approach to avoid this problem.\\
% Man könnt hier noch einen konkreten Sensorvorschlag machen, der auch mit ADC funktioniert 

\item
The last sensor to test was a photoresistor which is a semiconductor that reacts to light. It was also provided in the Arduino kit. The manual suggested the range of the photoresistor is between 500$\Omega$ (bright) and 50$k\Omega$ (dark). 
Further it suggested to use a 1$k\Omega$ resistor for the voltage divider. An arduino tutorial \citep{misc:photoresistor_tutorial} suggested 10 $k\Omega$. So we decided to measure the actual resistor values and use the formula from the lecture to calculate the matching resistor of the voltage divider. % , 26$k\Omega$ was a good choice  

The measurements showed that the actual photoresistor range is 500$\Omega$ to 1.7$M\Omega$.\\
\begin{equation}
R_1 = \sqrt{R_{max}*R_{max}} = \sqrt{500*1700000} = \sqrt{850000000} = 29k\Omega
\end{equation}


The closest resistor available to us was 26$k\Omega$.\\
These values differ to those suggested by the tutorials, but since also the photoresistor has different values, our choice seems feasible. Also we are convinced a higher resistor will not harm the components in contrast to a low resistor.\\
So we connected the chosen pulldown resistor between ground and the A1 pin of the Arduino Nano. To complete the voltage divider the photoresistor was connected between VCC and the A1 pin.\\

Sketches again are provided by online tutorials and the Arduino kit producer. Although it is very similar to the one for the moisture sensor, just another pin is used.\\

The test worked very well, close to a light bulb we get values over 950 and if we cover the photoresistor it drops to under 30. That confirmed our choice of the resistor.\\

We can conclude that this sensor is easy to work with and compatible with XBee, as no special protocols are required.

\end{itemize}

