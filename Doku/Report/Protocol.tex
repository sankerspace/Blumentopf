The application itself and project requirements bring certain constraints regarding the system architecture and protocol. For communication XBee, WiFi and RF24L01 chips have been considered. Because of costs, power and sensor constraints XBee and WiFi turned out to be not feasible.

\paragraph{Sensor Node <--> Controller}
One main constraint is that the sensor nodes should not be connected by cables. Therefore the communication with the controller has to be wireless and the node cannot be powered by the power grid.
For receiving messages on a wireless interface, the wireless module has to be actively listening continuously. The chosen modules RF24L01+ use about 15mA in listening mode. That makes a controller initiated communication impossible, as the constantly activated listening mode of the RF module would drain the batteries within 5 days.
To solve this problem, we chose that the sensor node initiates the transmission, waits for the answer of the controller and puts the RF module to a low power mode. This is an elegant solution for keeping the power down, however it is much more difficult to implement. It makes it necessary that every sensor is assigned a precise time at which it is expected to send its next measurement data to the controller in advance. Therefore the controller has the task of coordinating these timeslots and synchronizing the clocks of the sensor nodes. New sensor nodes should automatically be integrated into this schedule.


For registering nodes and perform data exchanges a protocol has been developed. The developed library provides access to the protocol registers to use the functionality of the protocol.


Status register in the node --> controller message:

\begin{table}[htbp]
        \small
        \setlength\tabcolsep{2pt}
\begin{tabular}{ |l|c|c|c|c|c|c|r| }
%  Test ID & measurement base frequency [MHz] & measurement scale factor & PLL factors (m/d) & measurement frequency [MHz] & PISO base frequency [MHz] & PISO scale factor & PISO frequency [MHz]\\ \hline
\hline
  Bit & name & description & 0 - meaning & 1 - meaning\\ \hline
  0 & RTC\_RUNNING\_BIT & RTC is paired on the node & no RTC & RTC paired \\ \hline
  1 & MSG\_TYPE\_BIT & type of message & registration request & data \\ \hline
  2 & NEW\_NODE\_BIT & new or known node & known node & new node \\ \hline
  3 & EEPROM\_DATA\_AVAILABLE & is data in the EEPROM & no data & data available\\ \hline
  4 & EEPROM\_DATA\_PACKED & kind of data & live data & EEPROM data\\ \hline
  5 & EEPROM\_DATA\_LAST & more EEPROM data pending & more data & not more data \\ \hline
  6 & NODE\_TYPE & type of node & sensor node & pump node\\ \hline
\end{tabular}\\
\end{table}


Status register in the controller --> node message:


\begin{table}[htbp]
        \small
        \setlength\tabcolsep{2pt}
\begin{tabular}{ |l|c|c|c|c|c|c|r| }
%  Test ID & measurement base frequency [MHz] & measurement scale factor & PLL factors (m/d) & measurement frequency [MHz] & PISO base frequency [MHz] & PISO scale factor & PISO frequency [MHz]\\ \hline
\hline
  Bit & name & description & 0 - meaning & 1 - meaning\\ \hline
  0 & REGISTER\_ACK\_BIT & registration acknowledge & no reg & registration\\ \hline
  1 & FETCH\_EEPROM\_DATA1 & &  &  \\ \hline
  2 & FETCH\_EEPROM\_DATA2 &  & &  \\ \hline
  3 & ID\_INEXISTENT & id invalid & id valid & id invalid\\ \hline
  7 & ID\_REGISTRATION\_ERROR &  & &  \\ \hline
\end{tabular}\\
\end{table}

Further the combination ``00'' in the FETCH\_EEPROM\_DATA1/2 bits indicates the EEPROM data shouldn't be transmitted now, ``01'' means the sensor node should send the data now and ``10'' means the node should delete the stored EEPROM data.
